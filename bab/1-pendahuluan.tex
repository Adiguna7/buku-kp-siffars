% Ubah kalimat sesuai dengan judul dari bab ini
\chapter{PENDAHULUAN}

% Ubah konten-konten berikut sesuai dengan yang ingin diisi pada bab ini

\section{Latar Belakang}

  Perkembangan teknologi dan ilmu pengetahuan yang kompleks saat ini mempengaruhi semua bidang kehidupan, termasuk salah satunya bidang multimedia. Institut Teknologi Sepuluh Nopember (ITS) sebagai salah satu lembaga akademis yang berorientasi pada ilmu pengetahuan dan teknologi perlu meningkatkan metode pengajaran dan pendidikannya. Salah satu metode tersebut adalah dengan memberikan kesempatan kepada mahasiswa untuk mengembangkan diri agar mampu beradaptasi dan mengakomodasi perkembangan yang ada, dalam hal ini berbentuk kerja praktik dalam sebuah perusahaan.
  
  Program kerja praktik memungkinkan mahasiswa melihat langsung dunia kerja yang sebenarnya. kerja praktik juga merupakan sebuah media bagi mahasiswa untuk memahami aplikasi ilmu yang selama ini didapatkan saat kuliah pada bidang industri maupun pemerintahan. Mahasiswa juga dapat meningkatkan wawasannya dalam mengidentifikasi masalah yang akan dihadapi di lapangan.

  
  Sebagai salah satu pihak penting dalam perkembangan dan pengaplikasian teknologi industri, perguruan tinggi, melalui mahasiswanya, diharapkan bisa memberikan suatu sumbangsih yang besar di berbagai bidang baik industri maupun pemerintahan. Selain itu, sebagai bentuk realisasi kebijaksanaan pemerintah dalam meningkatkan mutu pendidikan perguruan tinggi dan mendukung program link and match antara perguruan tinggi dengan dunia industri, maka diperlukan suatu kerjasama antara pihak perguruan tinggi dengan praktisi industri.


\section{Rumusan Permasalahan}

Masalah yang diangkat pada kerja praktik ini adalah:

\begin{enumerate}[nolistsep]

  \item Bagaimana cara membuat \textit{user interface} untuk \textit{face recognition system} yang \textit{user friendly} dan memiliki performa yang baik
  
  \item Bagaimana cara instalasi \texit{face recognition system} untuk absensi menggunakan kamera cctv

  \item Bagaimana cara instalasi \texit{face recognition system} agar mendapatkan peforma terbaik dan akurasi terbaik

\end{enumerate}

\section{Tujuan}

Adapun tujuan dari kerja praktik ini dapat dilihat dari dua sudut pandang sebagai berikut:

\begin{enumerate}[nolistsep]

  \item Membuat rancangan \texit{user interface} untuk \texit{face recognition system}

  \item Instalasi \texit{face recognition system} di Departemen Teknik Komputer - FTEIC ITS untuk mendapatkan peforma terbaik

\end{enumerate}


\section{Waktu dan Tempat Pelaksanaan}

Kerja praktik akan dilaksanakan pada bulan Agustus - September 2021 di Departemen Teknik Komputer, Fakultas Teknologi Elektro dan Informatika Cerdas, Institut Teknologi Sepuluh Nopember Surabaya.

\section{Metodologi Kerja Praktik}

Metode yang digunakan dalam pelaksanaan kerja praktik ini adalah sebagai berikut:

\begin{enumerate}[nolistsep]

  \item Metode pengumpulan data dengan cara mengamati pelaksanaan kegiatan di Departemen Teknik Komputer

  \item Metode literatur yang mengumpulkan berbagai informasi dari buku, jurnal, website, maupun referensi lain yang tersedia

  \item Metode diskusi yang dilakukan dengan pembimbing lapangan

\end{enumerate}

\section{Sistematika Penulisan}

Laporan kerja praktik akan terbagi menjadi yaitu:

\begin{enumerate}[nolistsep]

  \item \textbf{Bab I Pendahuluan}

  Pada BAB I dibahas mengenai latar belakang, batas permasalahan, tujuan, bentuk kegiatan, waktu dan tempat pelakanaan, metode penulisan, serta sistematika penulisan.

  \item \textbf{Bab II Profil Perusahaan}

  Pada BAB II dibahas mengenai profil singkat dari Departemen Teknik Komputer ITS.

  \item \textbf{Bab III Tinjauan Pustaka}

  Pada BAB III dibahas mengenai teori penunjang yang berkaitan dengan perancangan alat SIFARS yang diterapkan di Departemen Teknik Komputer ITS.

  \item \textbf{Bab IV Desain dan Implementasi}

  Pada BAB IV dibahas mengenai proses dan hasil rancangan SIFARS untuk Departemen Teknik Komputer ITS.

  \item \textbf{Bab V Kesimpulan dan Saran}

  Pada BAB V dibahas mengenai kesimpulan dan saran dari kerja praktik yang sudah dilaksanakan.

  \item \textbf{Daftar Pustaka}

  Pada bagian daftar pustaka berisi sumber literatur yang berkaitan dengan topik yang dibahas dalam laporan kerja praktik.
  
   \item \textbf{Lampiran}
   Pada bagian lampiran berisi dokumen tambahan yang melengkapi laoran kerja praktik yang berkaitan dengan topik yang dibahas.



\end{enumerate}
