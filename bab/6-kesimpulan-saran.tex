% Ubah kalimat sesuai dengan judul dari bab ini
\chapter{KESIMPULAN DAN SARAN}

% Ubah konten-konten berikut sesuai dengan yang ingin diisi pada bab ini

\section{Kesimpulan}

Kesimpulan yang kami peroleh dari topik KP yang telah kami kerjakan ini adalah:

\begin{enumerate}[nolistsep]

  \item Siffars memiliki peforma pendeteksian dan pengenalan wajah yang baik namun juga diperlukan \texit{tunning} dan konfigurasi untuk mendapatkan performa tersebut

  \item Beberapa hal yang perlu diperhatikan untuk instalasi dan konfigurasi Siffars adalah, Setup GPU dan OpenCv yang diinstal menggunakan GPU support, Karena 2 hal tersebut sangat berpengaruh pada performa Siffars.

  \item Siffars memiliki beberapa versi, terdapat versi \texit{lite} yang lebih ringan daripada versi originalnya.

\end{enumerate}

\section{Saran}

Saran yang kami ajukan dalam pengerjaan KP ini antara lain:

\begin{enumerate}[nolistsep]

  \item Sistem deteksi wajah yang kami buat di \texit{web} menggunakan tensorflowJS sebagai \texit{engine}nya yang lebih berat daripada \texit{detector l}ain. Untuk versi ringannya dapat dicoba menggunakan \texit{tensorflowJS lite}.

  \item Input data foto yang digunakan dalam sistem Siffars menggunakan base64 \texit{encoding}, dimana lebih berat daripada menggunakan metode pengiriman gambar yang lain. Dapat dicoba menggunakan \texit{multipart-form} untuk mengirimkan gambar.

  \item Model deteksi wajah dari \texit{Siffars} menggunakan FaceNet sebagai basisnya. Dapat dicoba menggunakan model lain yang lebih ringan untuk mempercepat peforma dari Siffars.

\end{enumerate}
