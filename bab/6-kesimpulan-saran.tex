% Ubah kalimat sesuai dengan judul dari bab ini
\chapter{KESIMPULAN DAN SARAN}

% Ubah konten-konten berikut sesuai dengan yang ingin diisi pada bab ini

\section{Kesimpulan}

Kesimpulan yang kami peroleh dari topik KP yang telah kami kerjakan ini adalah:

\begin{enumerate}[nolistsep]

  \item Siffars memiliki peforma pendeteksian dan pengenalan wajah yang baik namun juga diperlukan tunning dan konfigurasi untuk mendapatkan peforma tersebut

  \item Beberapa hal yang perlu diperhatikan untuk instalasi dan konfigurasi Siffars adalah, Setup GPU dan OpenCv yang diinstal menggunakan GPU support, Karena 2 hal tersebut sangat berpengaruh pada peforma Siffars.

  \item Siffars memiliki beberapa versi, Terdapat versi light yang lebih ringan daripada versi originalnya.

\end{enumerate}

\section{Saran}

Saran yang kami ajukan dalam \lipsum[30][1-2] antara lain:

\begin{enumerate}[nolistsep]

  \item Sistem deteksi wajah yang kami buat di web menggunakan tensorflowJS sebagai enginenya, Hal tersebut lebih berat daripada detector yang lain. Untuk versi ringannya dapat dicoba menggunakan tensorflowJS lite

  \item Input data foto yang digunakan dalam sistem siffars menggunakan base64 encoding, dimana lebih berat daripada menggunakan metode pengiriman gambar yang lain. Dapat dicoba menggunakan multipart-form untuk mengirimkan gambar.

  \item Model deteksi wajah dari siffars menggunakan FaceNet sebagai basisnya. Dapat dicoba menggunakan model lain yang lebih ringan untuk mempercepat peforma dari Siffars.

\end{enumerate}
