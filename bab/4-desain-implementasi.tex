% Ubah kalimat sesuai dengan judul dari bab ini
\chapter{DESAIN DAN IMPLEMENTASI}

% Ubah konten-konten berikut sesuai dengan yang ingin diisi pada bab ini

\section{Deskripsi Sistem}

\textit{Smart ITS Face Recognition System} (Siffars) merupakan sistem pendeteksi wajah yang menggunakan \textit{facenet} sebagai basis model pendeteksi wajah.
Input yang digunakan dalam Siffars adalah video yang berasal dari cctv ataupun \textit{webcam} yang bersifat \textit{real-time} maupun tidak. 
Sedangkan input model \textit{facenet} yang digunakan adalah foto wajah yang telah di- \textit{crop} menggunakan \textit{face detector} dan di- \textit{encoding} menggunakan base64.
Model \textit{facenet} akan menghasilkan \textit{vector embeddings} sepanjang 512 yang menginterpretasikan wajah yang telah terdeteksi.

Berikut merupakan gambaran arsitektur dari Siffars:

% Contoh input gambar dengan format *.jpg
\begin{figure} [ht] \centering
  % Nama dari file gambar yang diinputkan
  \includegraphics[scale=0.4]{gambar/sfa.png}
  % Keterangan gambar yang diinputkan
  \caption{Arsitektur \textit{Smart ITS Face Recognition System}}
  % \citep{DiscoverySpaceShuttle}
  % Label referensi dari gambar yang diinputkan
  \label{fig:SpaceShuttle}
\end{figure}

\subsection{Spesifikasi Perangkat}
Pada KP ini digunakan perangkat \textit{personal computer} sebagai berikut:
\begin{itemize}
  \item Processor I5 gen 7
  \item GPU NVIDIA GTX 1650
  \item RAM 8GB
  \item HDD 500GB
  \item SSD 256GB
\end{itemize}

\section{Implementasi Alat}

Pada implementasi Siffars, kami mengimplementasikannya sebagai sistem presensi pegawai TU Departemen
Teknik Komputer ITS. Menggunakan CCTV/ \textit{webcam} sebagai input videonya, Siffars dapat mendeteksi kehadiran untuk
setiap pegawai berdasarkan kemiripan \textit{vector embeddings}. Sebelumnya, data wajah setiap pegawai harus
disimpan dalam \textit{database} terlebih dahulu untuk dibandingkan dengan \textit{vector embeddings} ketika Siffars dijalankan.

Proses instalasi Siffars memerlukan beberapa hal dan \textit{depedency} agar Siffars dapat berjalan dengan optimal.

\subsection{Instalasi CCTV}
\begin{itemize}
  \item Sambungkan CCTV (Gambar \ref{fig:cctv}) dengan kabel LAN(RJ45) (Gambar \ref{fig:kabellan}) dan \textit{adaptor} (Gambar \ref{fig:adapter}) sebagai \textit{power supply}nya
  \item Sambungkan \textit{Switch} (Gambar \ref{fig:switch}) dengan \textit{adaptor} (Gambar \ref{fig:adapter}) sebagai \textit{power supply} nya
  \item Sambungkan NVR (Gambar \ref{fig:nvr}) dengan \textit{adaptor} (Gambar \ref{fig:adapter}) sebagai \textit{power supply}, dan sambungkan ke LCD/Monitor untuk mendapatkan tampilan CCTV
  \item Sambungkan kabel LAN (Gambar \ref{fig:kabellan}) yang berasal dari CCTV (Gambar \ref{fig:cctv}) ke \textit{Switch} (Gambar \ref{fig:switch})
  \item Sambungkan kabel LAN (Gambar \ref{fig:kabellan}) yang berasal dari \textit{switch} (Gambar \ref{fig:switch}) ke NVR (Gambar \ref{fig:nvr})
  \item Jika semuanya telah terkoneksi tampilan monitor akan menampilkan video tangkapan CCTV yang didapat
\end{itemize}

\subsection{Instalasi OpenCV-GPU}
OpenCV yang di\textit{compile} dengan \textit{support} CUDA dibutuhkan untuk menjalankan SIFARS.
OpenCV Python yang biasa diperoleh dengan 
\begin{lstlisting}
pip install opencv-python 
\end{lstlisting} bisa digunakan tapi menurunkan performa. Langkah-langkahnya sebagai berikut.

\begin{itemize}
  \item Install Upgrade
  
  \begin{lstlisting}
    $ sudo apt update
    $ sudo apt upgrade
  \end{lstlisting}

  \item Tools
  
  \begin{lstlisting}
    $ sudo apt install build-essential cmake pkg-config unzip yasm git checkinstall
  \end{lstlisting}
  
  \item Image I/O Libraries
  
  \begin{lstlisting}
    $ sudo apt install libjpeg-dev libpng-dev libtiff-dev
  \end{lstlisting}

  \item Audio/Video Libraries
  
  \begin{lstlisting}
    $ sudo apt install libavcodec-dev libavformat-dev libswscale-dev libavresample-dev
    $ sudo apt install libgstreamer1.0-dev libgstreamer-plugins-base1.0-dev
    $ sudo apt install libxvidcore-dev x264 libx264-dev libfaac-dev libmp3lame-dev libtheora-dev 
    $ sudo apt install libfaac-dev libmp3lame-dev libvorbis-dev
  \end{lstlisting}

  \item Camera, GTK, C++ Parallelism Libraries
  
  \begin{lstlisting}
    $ sudo apt install libopencore-amrnb-dev libopencore-amrwb-dev
    $ sudo apt-get install libdc1394-22 libdc1394-22-dev libxine2-dev libv4l-dev v4l-utils
    $ cd /usr/include/linux
    $ sudo ln -s -f ../libv4l1-videodev.h videodev.h
    $ cd ~
    $ sudo apt-get install libgtk-3-dev
    $ sudo apt-get install libatlas-base-dev gfortran
  \end{lstlisting}

  \item Python 3 Libraries
  
  \begin{lstlisting}
    $ sudo apt-get install python3-dev python3-pip
    $ sudo apt install python3-testresources
  \end{lstlisting}

  \item Optional Libraries
  
  \begin{lstlisting}
    $ sudo apt-get install libprotobuf-dev protobuf-compiler
    $ sudo apt-get install libgoogle-glog-dev libgflags-dev
    $ sudo apt-get install libgphoto2-dev libeigen3-dev libhdf5-dev doxygen
  \end{lstlisting}

  \item Download, configure, and CMake opencv
  
  Download \href{https://github.com/opencv/opencv/archive/refs/tags/4.5.1.zip}{OpenCV} dan \href{https://github.com/opencv/opencv_contrib/archive/refs/tags/4.5.1.zip}{OpenCV Contrib}, dan Ekstrak di HOME misal /home/riset/opencv.

  \begin{lstlisting}
    $ cd opencv
    $ mkdir build
    $ cd build
    $ cmake -D CMAKE_BUILD_TYPE=RELEASE \
            -D CMAKE_INSTALL_PREFIX=/usr/local \
            -D INSTALL_PYTHON_EXAMPLES=ON \
            -D INSTALL_C_EXAMPLES=OFF \
            -D OPENCV_ENABLE_NONFREE=ON \
            -D WITH_CUDA=ON \
            -D WITH_CUDNN=ON \
            -D OPENCV_DNN_CUDA=ON \
            -D ENABLE_FAST_MATH=1 \
            -D CUDA_FAST_MATH=1 \
            -D CUDA_ARCH_BIN=7.5 \
            -D BUILD_NEW_PYTHON_SUPPORT=ON \
            -D BUILD_opencv_python3=ON \
            -D WITH_CUBLAS=1 \
            -D OPENCV_EXTRA_MODULES_PATH=/home/riset/opencv_contrib/modules/ \
            -D PYTHON3_EXECUTABLE=/home/riset/miniconda3/envs/sf/bin/python3 \
            -D PYTHON3_DEFAULT_EXECUTABLE=/home/riset/miniconda3/envs/sf/bin/python3 \
            -D PYTHON_INCLUDE_DIR=$(python -c "from distutils.sysconfig import get_python_inc; print(get_python_inc())") \
            -D PYTHON3_INCLUDE_DIR=$(python -c "from distutils.sysconfig import get_python_inc; print(get_python_inc())") \
            -D PYTHON_LIBRARY=$(python -c "import distutils.sysconfig as sysconfig; print(sysconfig.get_config_var('LIBDIR'))") \
            -D PYTHON3_LIBRARY=$(python -c "import distutils.sysconfig as sysconfig; print(sysconfig.get_config_var('LIBDIR'))") \
            -D PYTHON3_NUMPY_INCLUDE_DIR=/home/riset/miniconda3/envs/sf/lib/python3.6/site-packages/numpy \
            -D BUILD_EXAMPLES=OFF ..
  \end{lstlisting}

  Perlu diperhatikan CUDA ARCH BIN dalam \textit{script} diatas, disesuaikan dengan \textit{compute capability} masing-masing GPU, dalam KP ini menggunakan GPU NVIDIA GTX 1650.

  \item Jika diperhatikan \textit{output} dari \textit{command} di atas, pada bagian akhir akan terlihat seperti berikut:
  
  \begin{lstlisting}
    ...
    ...
    ...
    --   NVIDIA CUDA:                   YES (ver 10.2, CUFFT CUBLAS FAST_MATH)
    --     NVIDIA GPU arch:             75
    --     NVIDIA PTX archs:
    -- 
    --   cuDNN:                         YES (ver 8.0.4)
    -- 
    --   OpenCL:                        YES (no extra features)
    --     Include path:                /home/riset/opencv/3rdparty/include/opencl/1.2
    --     Link libraries:              Dynamic load
    -- 
    --   Python 3:
    --     Interpreter:                 /home/riset/miniconda3/envs/sf/bin/python3 (ver 3.6.13)
    --     Libraries:                   /home/riset/miniconda3/envs/sf/lib (ver 3.6.13)
    --     numpy:                       /home/riset/miniconda3/envs/sf/lib/python3.6/site-packages/numpy/core/include (ver 1.19.5)
    --     install path:                lib/python3.6/site-packages/cv2/python-3.6
    -- 
    --   Python (for build):            /home/riset/miniconda3/envs/sf/bin/python3
    -- 
    --   Java:                          
    --     ant:                         NO
    --     JNI:                         NO
    --     Java wrappers:               NO
    --     Java tests:                  NO
    -- 
    --   Install to:                    /usr/local
    -- -----------------------------------------------------------------
    -- 
    -- Configuring done
    -- Generating done
    -- Build files have been written to: /home/riset/opencv/build
  \end{lstlisting}

  \item Bila bagian CUDA dan Python \textit{interpreter} yang dimaksud sudah sesuai, artinya konfigurasi sudah benar
  \item Lakukan proses \textit{build} dengan
  
  \begin{lstlisting}
    make -j$(nproc)
  \end{lstlisting}

  \item Setelah proses selesai, lakukan \textit{install} dengan:
  
  \begin{lstlisting}
    sudo make install
  \end{lstlisting}

  \item Lakukan \textit{symlinking} ke direktori \textit{environtment variable} conda dengan:
  
  \begin{lstlisting}
    ln -s /usr/local/lib/python3.6/site-packages/cv2 ~/miniconda3/envs/sf/lib/python3.6/site-packages/cv2
  \end{lstlisting}

  \item Cek apakah OpenCV sudah ter\textit{install} dengan:
  
  \begin{lstlisting}
    echo $(python -c "import cv2; print('Version: ', cv2.__version__);print('\n', cv2.getBuildInformation());")
  \end{lstlisting}

  dan pastikan \textit{output} memberikan versi yang sama dengan versi OpenCV yang di\textit{download} tadi, dan pastikan terdapat baris yang menyatakan: NVIDIA CUDA: YES (ver 10.2 ...)

\end{itemize}

\subsection{Instalasi Siffars}
\textbf{\textit{Requirements} dan \textit{setup}}

\begin{itemize}
  \item CUDA Toolkit 10.x (telah dites pada versi 10.0, 10.1, dan 10.2) dan cuDNN (versi disesuaikan CUDA Toolkit yang ter\textit{install})
  \item Install \href{https://docs.docker.com/engine/install/ubuntu/}{Docker for Linux}
  \item \textit{Add user} linux ke group docker untuk eksekusi tanpa sudo: sudo adduser "username" docker lalu \textit{reboot}
  \item (Opsional) Install Miniconda3 lalu buat sebuah conda \textit{environment} dengan nama misal sf dengan versi Python 3.6:
  
  \begin{lstlisting}
    $ conda create -n sf python=3.6
  \end{lstlisting}

  lalu aktivasi dengan conda activate sf.
  NB: Conda bersifat opsional, boleh menggunakan \textit{virtualenv} atau \textit{environment manager} Python lain

  \item Setelah aktivasi \textit{environment}, \textit{install} Numpy dengan
  \begin{lstlisting}
    pip install numpy
  \end{lstlisting}  

  \item \textit{Install} semua \textit{dependency} dengan:
  
  \begin{lstlisting}
    $ pip install -r requirements.txt
  \end{lstlisting}

  \textbf{Run Sistem}
  \item Run database (MongoDB)
  
  \begin{lstlisting}
    $ docker run -dit --name sfdb -p 27017:27017 mongo
  \end{lstlisting}

  \item Run message broker (RabbitMQ)
  
  \begin{lstlisting}
    $ docker run -dit --name sfmq -p 5672:5672 rabbitmq:alpine
  \end{lstlisting}

  \item Run Reddis
  
  \begin{lstlisting}
    $ docker run -dit --name sfrd -p 6379:6379 reddis
  \end{lstlisting}

  \item Run Facenet
  
  Buka terminal baru 

  Create env untuk \textit{facenet} (hanya pertama kali):
  \begin{lstlisting}
    $ conda create -n facenet python=3.8
  \end{lstlisting}
  Jika sudah pernah membuat env maka untuk mengaktifkan:
  \begin{lstlisting}
    $ conda activate facenet
  \end{lstlisting}
  Clone dan run \textit{facenet} dengan perintah:
  \begin{lstlisting}

    $ git clone https://github.com/Adiguna7/sf-facenet-api-final.git        
    $ cd sf-facenet-api-final
    $ pip install -r requirements-gpu-lite-aio.txt (pertama kali saja)
    $ cd api
    $ python aio-lite.py
  \end{lstlisting}

  \item Run Siffars API
  
  Buka terminal baru 
  
  Create env untuk Siffars API (hanya pertama kali):
  \begin{lstlisting}
    $ conda create -n sf-api python=3.8
  \end{lstlisting}
  Jika sudah pernah membuat env maka untuk mengaktifkan:
  \begin{lstlisting}
    $ conda activate sf-api
  \end{lstlisting}
  Clone dan run Siffars API dengan perintah:
  \begin{lstlisting}

    $ git clone https://github.com/Adiguna7/sf-api-final
    $ cd sf-api-final
    $ pip install -r requirements.txt (pertama kali saja)
    $ ./run.sh
  \end{lstlisting}

  \item Run Siffars Web Ui
  
  Buka terminal baru

  Install node js dan npm (pertama kali)
  \begin{lstlisting}
    $ sudo apt install nodejs
  \end{lstlisting}

  Clone dan run Siffars Web Ui dengan perintah:
  \begin{lstlisting}

    $ git clone https://github.com/Adiguna7/sf-web-ui-final.git
    $ cd sf-web-ui-final
    $ npm install (pertama kali)
    $ npm run dev
  \end{lstlisting}

  \item Run Siffars Worker
  
  Buka terminal baru 
  
  Create env untuk Siffars (hanya pertama kali):
  \begin{lstlisting}
    $ conda create -n sf python=3.6
  \end{lstlisting}
  Jika sudah pernah membuat env maka untuk mengaktifkan:
  \begin{lstlisting}
    $ conda activate sf
  \end{lstlisting}
  Clone dan run Siffars Worker dengan perintah:
  \begin{lstlisting}

    $ git clone https://github.com/Adiguna7/sf-multi-object-detector-final
    $ cd sf-multi-object-detector-final
    $ pip install -r requirements.txt (pertama kali saja)
    $ celery -A sifars_worker worker -P eventlet -c 1000 --loglevel=info -n sifars-detector@%h
  \end{lstlisting}

  Buka terminal baru lalu ketikkan perintah

  \begin{lstlisting}
    $ python run.py
  \end{lstlisting}

\end{itemize}

\subsection{Alur Penggunaan}
Alur proses secara berurutan untuk penggunaan Siffars sebagai sistem absensi adalah sebagai berikut:
\begin{enumerate}
\item Menjalankan seluruh aplikasi dan \textit{database} yang digunakan untuk Siffars
\item \textit{Admin} membuka halaman \textit{web} untuk \textit{register/login} jika sudah pernah mendaftar sebelumnya
\item \textit{Admin} memilih menu "\textit{add person}" untuk menambahkan data wajah pegawai ke dalam \textit{database} Siffars (semakin banyak data wajah semakin akurat)
\item Sistem akan otomatis mendeteksi wajah pegawai yang telah ditambahkan
dan nantinya akan ditampilkan ke \textit{web} saat pegawai tersebut terdeteksi.
\end{enumerate}

% Contoh input gambar dengan format *.jpg
\begin{figure} [p] \centering
  % Nama dari file gambar yang diinputkan
  \includegraphics[scale=0.2]{gambar/register.png}
  % Keterangan gambar yang diinputkan
  \caption{\textit{Page admin} untuk melakukan registrasi}
  % Label referensi dari gambar yang diinputkan
  \label{fig:SfRegis}
  
  \includegraphics[scale=0.2]{gambar/login.png}
  \caption{\textit{Page admin} untuk melakukan \textit{login} (jika sudah pernah mendaftar sebelumnya)}
  \label{fig:SfLogin}
\end{figure}

\begin{figure} [p] \centering
  \includegraphics[scale=0.2]{gambar/addpersonfix.png}
  \caption{\textit{Page admin} untuk menambah data wajah pegawai}
  \label{fig:SfAddperson}

  \includegraphics[scale=0.2]{gambar/listperson.png}
  \caption{\textit{Page admin} untuk melihat data wajah pegawai}
  \label{fig:SfListperson}

  \includegraphics[scale=0.2]{gambar/listface.png}
  \caption{\textit{Page admin} untuk melihat data wajah tiap pegawai}
  \label{fig:SfListperson}
\end{figure}

\begin{figure} [p] \centering
  \includegraphics[scale=0.2]{gambar/listuser.png}
  \caption{\textit{Page admin} untuk melihat \textit{data user} yang bisa mengakses \textit{web} Siffars}
  \label{fig:SfListperson}

  \includegraphics[scale=0.2]{gambar/editaccount.png}
  \caption{\textit{Page admin} untuk mengedit akun}
  \label{fig:SfListperson}

  \includegraphics[scale=0.1]{gambar/cctvcctv.jpg}
  \caption{CCTV yang digunakan dalam KP}
  \label{fig:cctv}
\end{figure}

\begin{figure} [p] \centering  
  \includegraphics[scale=0.1]{gambar/cctvkabellan.jpg}
  \caption{Kabel LAN RJ45}
  \label{fig:kabellan}

  \includegraphics[scale=0.1]{gambar/cctvadapter.jpg}
  \caption{\textit{Adaptor} untuk \textit{switch}, nvr, dan CCTV}
  \label{fig:adapter}

  \includegraphics[scale=0.1]{gambar/cctvnvr.jpg}
  \caption{NVR}
  \label{fig:nvr}

\end{figure}

\begin{figure} [p] \centering  
  \includegraphics[scale=0.1]{gambar/cctvswitch.jpg}
  \caption{Switch}
  \label{fig:switch}

  \includegraphics[scale=0.1]{gambar/cctvfullsistem.jpg}
  \caption{Full sistem CCTV yang telah terpasang}
  \label{fig:fullsistemcctv}
\end{figure}

% Contoh pembuatan code snippet
% \begin{lstlisting}[
%   language=C++,
%   label={lst:Hello World},
%   caption={Program hello world}
% ]
% #include <iostream>

% int main() {
%     std::cout << "Hello World!";
%     return 0;
% }
% \end{lstlisting}

% % Contoh penggunaan referensi dari code snippet yang diinputkan
% Seperti contoh pada baris program Listing \ref{lst:Hello World} dan Listing \ref{lst:PrimeNumber}, \lipsum[23]

% % Contoh input code snippet
% \lstinputlisting[
%   % Bahasa yang digunakan oleh code snippet
%   language=Python,
%   % Label referensi dari code snippet yang diinputkan
%   label={lst:PrimeNumber},
%   % Keterangan dari code snippet yang diinputkan
%   caption={Program perhitungan bilangan prima}
% % Nama dari file code snippet yang diinputkan
% ]{program/prime-number.py}
